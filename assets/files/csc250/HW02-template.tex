% --------------------------------------------------------------
% This is all preamble stuff that you don't have to worry about.
% Head down to where it says "Start here"
% --------------------------------------------------------------
 
\documentclass[12pt]{article}
 
\usepackage[margin=1in]{geometry} 
\usepackage{amsmath,amsthm,amssymb}
\usepackage{graphicx}
\usepackage{enumerate}
\usepackage{xcolor}
\definecolor{smithblue}{HTML}{002855}
\definecolor{smithyellow}{HTML}{F2A900}

\usepackage[parfill]{parskip}
\parskip=\baselineskip
 
\newcommand{\N}{\mathbb{N}}
\newcommand{\Z}{\mathbb{Z}}
 

\newenvironment{exercise}[2][Exercise]{\begin{trivlist}
\item[\hskip \labelsep {\bfseries #1}\hskip \labelsep {\bfseries #2.}]}{\end{trivlist}}

\newenvironment{solution}[1][{\color{red} Solution:}]{\begin{trivlist}
\item[\hskip \labelsep {\bfseries #1}\hskip \labelsep {\bfseries}]}{\end{trivlist}}


\usepackage{fancyhdr}
\pagestyle{fancy}
\lhead{Submitted by: \studentName\\
\collaborators}
\rhead{CSC250 Fall 2025 - Homework 02\\
\today{}}
\cfoot{p. \thepage}
\renewcommand{\headrulewidth}{0.4pt}
\renewcommand{\footrulewidth}{0.4pt}

\begin{document}
 
% --------------------------------------------------------------
%                         Start here
% --------------------------------------------------------------


\newcommand{\studentName}{YOUR NAME HERE} %replace with your name

\newcommand{\collaborators}{
	% Comment out the line below if you worked alone
	with \textit{COLLABORATORS' NAMES HERE}
	% Uncomment the line below if you worked alone
	% \textit{I did not collaborate with anyone on this assignment.}
}

% --------------
% Exercise 1
% --------------
\begin{exercise}{1}
For each of the following REs on the alphabet $\Sigma = [a,b,c]$, identify one word that is \textbf{in the language} recognized by the RE, and one word that is \textbf{not in the language}.
\begin{enumerate}[(a)]
\item $(ab)^*+bc^*$
% -------------------------------------------
%  Write your answer to Q1a below
% -------------------------------------------
\begin{solution} 
\quad\\
Your solution Here
\end{solution}

\item $a^*bc^*$
% -------------------------------------------
%  Write your answer to Q1b below
% -------------------------------------------
\begin{solution}
\quad\\
Your solution Here
\end{solution}

\item $(a+b)^*(b+c)^*$
% -------------------------------------------
%  Write your answer to Q1c below
% -------------------------------------------
\begin{solution}
\quad\\
Your solution Here 
\end{solution}

\end{enumerate}
\end{exercise}

\clearpage

% --------------
% Exercise 2
% --------------
\begin{exercise}{2}

Give a regular expression for each of the following languages. In all cases, the alphabet is $\Sigma = \{0,1\}$.
\begin{enumerate}[(a)]
	\item $\{ w \in \Sigma^* \ | \ w \texttt{ has an odd number of 0s followed by a single 1}\}$
	% -------------------------------------------
	%  Write your answer to Q2a below
	% -------------------------------------------
	\begin{solution}
Your solution Here
	\end{solution}
	
	\item $\{w \in \Sigma^* \ | \ w \texttt{ starts in a triple letter (000 or 111)}\}$
	% -------------------------------------------
	%  Write your answer to Q2b below
	% -------------------------------------------
	\begin{solution}
Your solution Here
	\end{solution}
	
	\item $\{w \in \Sigma^* \ | \ w \texttt{ contains exactly three 1s or at least two 0s}\}$
	% -------------------------------------------
	%  Write your answer to Q2c below
	% -------------------------------------------
	\begin{solution}
Your solution Here
	\end{solution}
	
\end{enumerate}

\end{exercise}

\clearpage
% --------------
% Exercise 3
% --------------
\begin{exercise}{3}
Show that any language $L_F$ containing only finitely many strings is regular.
\end{exercise}

% -------------------------------------------
%  Write your answer to Q3 below
% -------------------------------------------
\begin{solution}

HINT: To prove $L_F$ is regular, we simply need to show there is a regular expression to generate it;

\begin{proof}[\unskip\nopunct]
Your solution Here
\end{proof}

\end{solution}


% \medskip


% % --------------
% % Exercise 4
% % --------------
% \begin{exercise}{4}
% Show that, if $L_1$ and $L_2$ are any two regular languages, then $(L_1)^cL_2$ is also a regular language.

% $(L_1)^c$ means the complement of $L_1$; For example, if $L_1$ is the set of strings with $001$ as a substring, then $(L_1)^c$ is the set of strings that don't have $001$ as a substring.
% \end{exercise}

% % -------------------------------------------
% %  Write your answer to Q4 below
% % -------------------------------------------
% \begin{solution}
% \end{solution}

\clearpage

% --------------
% Exercise 5\4
% --------------
\begin{exercise}{4}

Show that if $L$ is a regular language, then the reverse language $L^R$:

$$L^R = \{w^R \ | \ w\in L \texttt{ and } w^R \texttt{ is the word } w \texttt{ written in reverse}\}$$

is also a regular language (i.e. regular languages are closed under \textbf{reversal}).

Remember, proving a language is regular is all about proving it has a regular expression that can generate it!

{\Large \textit{Hint: }}
 try induction! by:
\begin{enumerate}
    \item \textbf{Base cases}: prove this works for the 3 base cases:
    \begin{itemize}
        \item $L =\emptyset$ (show the reverse of an empty language is regular)
        \item $w=\epsilon$ (show you can get the regular expression for the reverse of an empty word)
        \item $w=a$ \text{ where} $a$ \text{ represents any single valid symbol}(show you can get the regular expression for the reverse of a word made of a single symbol)
    \end{itemize}
    \item\textbf{ Induction Hypothesis}: assume that there exist two reversible regular languages, $L_1 = L(R_1)$ and $L_2 = L(R_2)$ such that $L^R_1 = L(R^R_1)$ and $L^R_2 = L(R^R_2)$.

    \item \textbf{Induction Step}: Using the hypotheses, show that the reverse of any regular expression is also a regular expression (
    the reverse of regular expressions that use alternation, concatenation, and Kleene star are also regular expressions); 
\end{enumerate}

\end{exercise}

% -------------------------------------------
%  Write your answer to Q5 below
% -------------------------------------------
\begin{solution}
Your solution Here
\end{solution}


% -----------------
% References
% -----------------
\vfill
\begin{thebibliography}{9}
\bibitem{sipser} 
Sipser, Michael. 
\textit{Introduction to the Theory of Computation.}
Course Technology, 2005. ISBN: 9780534950972

\bibitem{critchlow2011foundation}
Critchlow, Carol and Eck, David
\textit{Foundation of computation.},
Critchlow Carol, 2011

\end{thebibliography}

% --------------------------------------------------------------
%     You don't have to mess with anything below this line.
% --------------------------------------------------------------
 
\end{document}