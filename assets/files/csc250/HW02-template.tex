% --------------------------------------------------------------
% This is all preamble stuff that you don't have to worry about.
% Head down to where it says "Start here"
% --------------------------------------------------------------
 
\documentclass[12pt]{article}
 
\usepackage[margin=1in]{geometry} 
\usepackage{amsmath,amsthm,amssymb}
\usepackage{graphicx}
\usepackage{enumerate}
\usepackage{xcolor}
\usepackage{lastpage}
\definecolor{smithblue}{HTML}{002855}
\definecolor{smithyellow}{HTML}{F2A900}
\usepackage[framemethod=TikZ]{mdframed}
\mdfsetup{%
backgroundcolor=gray!20,
roundcorner=4pt}
\newmdenv[frametitle={\textbf{\textit{Hint:}}}]{tip}

\usepackage[parfill]{parskip}
\parskip=\baselineskip
 
\newcommand{\N}{\mathbb{N}}
\newcommand{\Z}{\mathbb{Z}}
 
\newenvironment{exercise}[2][Exercise]{\begin{trivlist}
\item[\hskip \labelsep {\bfseries #1}\hskip \labelsep {\bfseries #2.}]}{\end{trivlist}}

\newenvironment{solution}[1][{\color{red} Solution:}]{\begin{trivlist}
\item[\hskip \labelsep {\bfseries #1}\hskip \labelsep {\bfseries}]}{\end{trivlist}}


\usepackage{fancyhdr}
\pagestyle{fancy}
\lhead{Submitted by: \mainName\\
\collaborators}
\rhead{CSC250 Fall 2024 - Homework 02\\
\today{}}
\cfoot{p. \thepage \ of \pageref{LastPage}}
\renewcommand{\headrulewidth}{0.4pt}
\renewcommand{\footrulewidth}{0.4pt}
 

%\newcommand\solution[1]{\vskip 5pt \noindent{\color{red}{\bf Solution:}} \emph{#1}}
 
\begin{document}
 
% --------------
% Exercise 1
% --------------
\begin{exercise}{1}
For each of the following REs on the alphabet $\Sigma = [a,b,c]$, identify one word that is \textbf{in the language} recognized by the RE, and one word that is \textbf{not in the language}.
\begin{enumerate}[(a)]
\item $(ab)^*+bc^*$
% -------------------------------------------
%  Write your answer to Q1a below
% -------------------------------------------
\begin{solution} 
\quad\\
Your solution Here
\end{solution}

\item $a^*bc^*$
% -------------------------------------------
%  Write your answer to Q1b below
% -------------------------------------------
\begin{solution}
\quad\\
Your solution Here
\end{solution}

\item $(a+b)^*(b+c)^*$
% -------------------------------------------
%  Write your answer to Q1c below
% -------------------------------------------
\begin{solution}
\quad\\
Your solution Here 
\end{solution}

\end{enumerate}
\end{exercise}

\clearpage

% --------------
% Exercise 2
% --------------
\begin{exercise}{2}

Give a regular expression for each of the following languages. In all cases, the alphabet is $\Sigma = \{0,1\}$.
\begin{enumerate}[(a)]
	\item $\{ w \in \Sigma^* \ | \ w \texttt{ has an odd number of 0s followed by a single 1}\}$
	% -------------------------------------------
	%  Write your answer to Q2a below
	% -------------------------------------------
	\begin{solution}
		Your solution here
	\end{solution}
	
	\item $\{w \in \Sigma^* \ | \ w \texttt{ starts in a triple letter (000 or 111)}\}$
	% -------------------------------------------
	%  Write your answer to Q2b below
	% -------------------------------------------
	\begin{solution}
		Your solution here
	\end{solution}
	
	\item $\{w \in \Sigma^* \ | \ w \texttt{ contains exactly three 1s or at least two 0s}\}$
	% -------------------------------------------
	%  Write your answer to Q2c below
	% -------------------------------------------
	\begin{solution}
		Your solution here
	\end{solution}
	
\end{enumerate}

\end{exercise}

\clearpage
% --------------
% Exercise 3
% --------------
\begin{exercise}{3}
Show that any language $L_F$ containing only finitely many strings is regular.
\end{exercise}

\begin{tip}
To prove that a language is regular, it is sufficient to show that there is a regular expression to generate it!
\end{tip}
% -------------------------------------------
%  Write your answer to Q3 below
% -------------------------------------------


\begin{solution}

\begin{proof}[\unskip\nopunct]
	Your solution here
\end{proof}

\end{solution}

\clearpage

% --------------
% Exercise 4
% --------------
\begin{exercise}{4}

Show that if $L$ is a regular language, then the reverse language $L^R$:

$$L^R = \{w^R \ | \ w\in L \texttt{ and } w^R \texttt{ is the word } w \texttt{ written in reverse}\}$$

is also a regular language (i.e. regular languages are closed under \textbf{reversal}).

Remember, proving a language is regular is the same as proving it has a regular expression that can generate it!

\begin{tip}
Try induction! by:
\begin{enumerate}
    \item \textbf{Base cases}: prove this works for the 3 base cases:
    \begin{itemize}
        \item $L =\emptyset$ (show the reverse of an empty language is regular)
        \item $w=\epsilon$ (show you can get the regular expression for the reverse of an empty word)
        \item $w=a$ \text{ where} $a$ \text{ represents any single valid symbol}(show you can get the regular expression for the reverse of a word made of a single symbol)
    \end{itemize}
    \item\textbf{ Induction Hypothesis}: assume that  
    \begin{itemize}
        \item the alternation of two regular expressions is also a regular expression
        \item the concatenation of two regular expressions is also a regular expression
        \item applying Klenee Star to a regular expression results in a regular expression
        % \item two regular expressions: $E$ and $F$ that generate $L_F = L(F)$ and $L_G = L(G)$ respectively, AND for which
        % \item we are able to find a regular expressions $F^R$ and $G^R$ such that  $L^R_F = L(F^R)$ and $L^R_G = L(G^R)$, and where $L^R_F = \text{ Reverse of } L_F$ and $L^R_G = \text{ Reverse of } L_G$
    \end{itemize}

    \item \textbf{Induction Step}: Using the hypotheses, show that the reverse of any regular expression is also a regular expression (
    the reverse of regular expressions using alternation, concatenation, and Kleene star are also regular expressions); 
\end{enumerate}
\end{tip}

\end{exercise}

% -------------------------------------------
%  Write your answer to Q4 below
% -------------------------------------------
\begin{solution}
Your solution here
\end{solution}


% -----------------
% References
% -----------------
\vfill
\begin{thebibliography}{9}
\bibitem{sipser} 
Sipser, Michael. 
\textit{Introduction to the Theory of Computation.}
Course Technology, 2005. ISBN: 9780534950972

\bibitem{critchlow2011foundation}
Critchlow, Carol and Eck, David
\textit{Foundation of computation.},
Critchlow Carol, 2011

\end{thebibliography}

% --------------------------------------------------------------
%     You don't have to mess with anything below this line.
% --------------------------------------------------------------
 
\end{document}