% --------------------------------------------------------------
% This is all preamble stuff that you don't have to worry about.
% Head down to where it says "Start here"
% --------------------------------------------------------------
 
\documentclass[12pt]{article}
 
\usepackage[margin=1in]{geometry} 
\usepackage{amsmath,amsthm,amssymb}
\usepackage{graphicx}
\usepackage{enumerate}
\usepackage{enumitem}
\usepackage{xcolor}
\definecolor{smithblue}{HTML}{002855}
\definecolor{smithyellow}{HTML}{F2A900}

\usepackage[parfill]{parskip}
\parskip=\baselineskip
 
\newcommand{\N}{\mathbb{N}}
\newcommand{\Z}{\mathbb{Z}}
 

\newenvironment{exercise}[2][Exercise]{\begin{trivlist}
\item[\hskip \labelsep {\bfseries #1}\hskip \labelsep {\bfseries #2.}]}{\end{trivlist}}

\newenvironment{solution}[1][{\color{red} Solution:}]{\begin{trivlist}
\item[\hskip \labelsep {\bfseries #1}\hskip \labelsep {\bfseries}]}{\end{trivlist}}


\usepackage{fancyhdr}
\pagestyle{fancy}
\lhead{Submitted by: \studentName\\
\collaborators}
\rhead{CSC250 Fall 2025 - Recovery Homework\\
\today{}}
\cfoot{p. \thepage}
\renewcommand{\headrulewidth}{0.4pt}
\renewcommand{\footrulewidth}{0.4pt}

\begin{document}
 
% --------------------------------------------------------------
%                         Start here
% --------------------------------------------------------------


\newcommand{\studentName}{YOUR NAME HERE} %replace with your name

\newcommand{\collaborators}{
	% Comment out the line below if you worked alone
	with \textit{COLLABORATORS' NAMES HERE}
	% Uncomment the line below if you worked alone
	% \textit{I did not collaborate with anyone on this assignment.}
}

% --------------
% Exercise 1
% --------------
\begin{exercise}{1}
Consider the following language:
\[COMPOSITE_n = \{n \ | \ n = ab \texttt{ for some integers } a,b\}\]
What is the smallest class that contains this language (finite, regular, context-free, decidable, recognizable, or unrecognizable)? Prove it.
\end{exercise}

% -------------------------------------------
%  Write your answer to Q1 below
% -------------------------------------------
\begin{solution}
Your solution here
\end{solution}

\clearpage

% --------------
% Exercise 2
% --------------
\begin{exercise}{2}
Consider the following language:
\begin{eqnarray*}
COMPOSITE_{TM} & = & \{\langle M,w \rangle \ | \ M \texttt{ is a TM and } M \texttt{ halts on } w \texttt{ in }\\
&& \ \ \ \ \ \ \ \ \ \ \ \ \ n = ab \texttt{ steps for some integers } a,b\}
\end{eqnarray*}
What is the smallest class that contains this language (finite, regular, context-free, decidable, recognizable, or unrecognizable)? Prove it.
\end{exercise}

% -------------------------------------------
%  Write your answer to Q2 below
% -------------------------------------------
\begin{solution}
Your solution here
\end{solution}

\clearpage
% --------------
% Exercise 3
% --------------
\begin{exercise}{3}
Consider the following language:
\[COMPOSITE_{RE} = \{n \ | \ n = ab \texttt{ for some regular expressions } a,b\}\]
What is the smallest class that contains this language (finite, regular, context-free, decidable, recognizable, or unrecognizable)? Prove it.
\end{exercise}

% -------------------------------------------
%  Write your answer to Q3 below
% -------------------------------------------
\begin{solution}
Your solution here
\end{solution}

\clearpage

% --------------
% Exercise 4
% --------------
\begin{exercise}{4}

Describe the primary differences between a Turing reduction ($\le_T$) and a Mapping reduction ($\le_m$).

\end{exercise}

% -------------------------------------------
%  Write your answer to Q4 below
% -------------------------------------------
\begin{solution}
Your solution here
\end{solution}

\clearpage


% \hrule
% \vskip 2em 
% --------------
% Exercise 1
% --------------
\begin{exercise}{5}
Show that the following language is in $P$:
\[\texttt{RELATIVELY-PRIME}=\{ \langle x,y \rangle \ | \ x \texttt{ and } y \texttt{ are integers, } gcd(x, y) = 1\}\]
\end{exercise}



\clearpage
% \vskip 1em 
% \hrule
% \vskip 1em 


% --------------
% Exercise 2
% --------------
\begin{exercise}{6}
A Caesar cipher is a simplified encryption protocol in which all letters are shifted $0 < k < 26$ positions $mod \ 26$, e.g. when $k=3$:\\\\
\includegraphics[width=\textwidth]{caesar.png}\\
To use this encryption method, look up the substitution for each letter, like this:
\begin{center}
{\Large SMITH COLLEGE$\rightarrow$ VPLWK FROOHJH}
\end{center}
Show that this encryption scheme can be broken in $O(n)$ where $n$ is the length of the message.
\end{exercise}

\clearpage
% \vskip 1em 
% \hrule
% \vskip 1em 

% --------------
% Exercise 3
% --------------
\begin{exercise}{7}
Consider the language: 
\begin{eqnarray*}
VERTEX-COVER &=& \{\langle G,k \rangle  \ | \ G \texttt{ is a graph that has a} \\
&&\texttt{ \ \ \ \ \ \ \ vertex cover of size }k\}
\end{eqnarray*}
where a \textbf{vertex cover} is a set of $k$ vertices such that every edge in the graph touches at least one of the vertices.

\begin{enumerate}[label=(\alph*)]
	\item Draw a diagram of a graph on 10 vertices with an \textbf{vertex cover} of size 5.
	\item Prove that $VERTEX-COVER$ is $NP$-complete.
\end{enumerate}
\end{exercise}

\clearpage
% \vskip 1em 
% \hrule
% \vskip 1em 


% --------------
% Exercise 4
% --------------
\begin{exercise}{8}
Consider the language: 
\begin{eqnarray*}
SET-COVER &=& \{\langle U,S,k \rangle  \ | \ U \texttt{ is a set of elements } \{1, 2, ..., n\} \texttt{ (the "universe"),} \\
&& \ \ \ \ \ \ \ \ \ \ \ \ \ \ \ S \texttt{ is a set of } m \texttt{ subsets where } \bigcup S = U, \\
&& \ \ \ \ \ \ \ \ \ \ \ \ \ \ \ \texttt{and } S \texttt{ contains a set cover of size } k\}
\end{eqnarray*}
where a \textbf{set cover} is a set of $k$ subsets $\in S$ such that every element in $U$ is contained in at least one of the selected subsets.\\

\begin{enumerate}[label=(\alph*)]
	\item Draw a diagram of a universe with 10 elements, partitioned into 5 subsets with a \textbf{set cover} of size 3.
	\item Prove that $SET-COVER$ is $NP$-complete.
\end{enumerate}
\end{exercise}




% -----------------
% References
% -----------------
\vfill
\begin{thebibliography}{9}
\bibitem{sipser} 
Sipser, Michael. 
\textit{Introduction to the Theory of Computation.}
Course Technology, 2005. ISBN: 9780534950972

\bibitem{critchlow2011foundation}
Critchlow, Carol and Eck, David
\textit{Foundation of computation.},
Critchlow Carol, 2011

\end{thebibliography}

% --------------------------------------------------------------
%     You don't have to mess with anything below this line.
% --------------------------------------------------------------
 
\end{document}