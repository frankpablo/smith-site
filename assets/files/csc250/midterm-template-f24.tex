% -------------------------------------------------------------------------------
% This is all preamble stuff that you don't have to worry about.
% Head down to where it says "Enter your name here"
% -------------------------------------------------------------------------------
 
\documentclass[12pt]{article}
 
\usepackage[margin=1in]{geometry} 
\usepackage{amsmath,amsthm,amssymb}
\usepackage{graphicx}
\usepackage{enumerate}
\usepackage[names]{xcolor}
\usepackage[multiple,hang,flushmargin]{footmisc}
\usepackage{lastpage}
\usepackage{tikz}


\usepackage[parfill]{parskip}
\parskip=\baselineskip

 
\newenvironment{question}[2][Question]{\begin{trivlist}
\item[\hskip \labelsep {\bfseries #1}\hskip \labelsep {\bfseries #2.}]}{\end{trivlist}}

\newenvironment{solution}[1][Solution:]{\begin{trivlist}
\item[\hskip \labelsep {\bfseries #1}\hskip \labelsep {\bfseries}]\color{blue}}{\end{trivlist}}

\usepackage{fancyhdr}
\pagestyle{fancy}
\rhead{Due: \duedate}
\lhead{CSC250 - Midterm}
\cfoot{p. \thepage \ of \pageref{LastPage}}
\renewcommand{\headrulewidth}{0.4pt}
\renewcommand{\footrulewidth}{0.4pt}

\begin{document}
\newcommand{\duedate}{11:59PM EST Thursday October 24, 2024}
 

% --------------------------------------------------------------------------------------------
%  Great, now skip ahead to where you see *** START EXAM HERE ***
% --------------------------------------------------------------------------------------------

\topskip0pt
\vspace*{\fill}
\begin{center}
\fbox{\fbox{\parbox{5.5in}{\begin{center}
\vspace{1em}
This is the first of two examinations for\\ \textbf{CSC250: Theory of Computation}\\ as taught by Pablo Frank Bolton in Fall 2024.\\
\vspace{1em}
This exam is \textbf{open book and open note}.\\ The following materials are \textbf{permitted} while taking this examination:
\vspace{1em}
\begin{itemize}
	\item your own notes
	\item lecture slides / videos
	\item scribe notes
	\item homework solutions
	\item any of the recommended textbooks
\end{itemize}
\vspace{1em}
\textbf{Honor code: no other resources are permitted during this exam.}\\
This includes (but is not limited to): online materials, tutors, teaching assistants, and other students.
\end{center}
\vspace{5em}
 \ \ \ YOUR NAME: \underline{\hspace{10cm}}
\vspace{3em}}}}
\end{center}
\vspace{3em}
\vspace*{\fill}


% ---------------------------------------
%   *** START EXAM HERE ***
% ---------------------------------------


\clearpage


% ---------------------------------------
%   Question 1: Propositional Logic (4 pnts) (4/40)
% ---------------------------------------
\begin{question}{1}\textbf{Valid or Invalid Reasoning} (4 points)\\\\
For each of the following English arguments, 
\begin{itemize}
    \item express the argument in terms of propositional logic, and
    \item briefly justify whether the argument is valid or invalid.
\end{itemize}
  Be sure to clearly label your propositions.
\begin{enumerate}[(a)]	

    \item A golf club has the following rule: \textit{ If a club member is found cheating or being disruptive, they are suspended for a week}; My friend Barney, who never cheats, was suspended for a week;  \textbf{therefore, Barney must have been disruptive}

    \begin{solution}
       Your solution here
    \end{solution}

    \item \textit{Show, \textbf{using a table}, if the following statement is true or false: "If I have an apple, I smile; I am not smiling, \textbf{therefore I did not have an apple}"}
    
    \begin{solution}
       Your solution here
    \end{solution}
    
\end{enumerate}
\end{question}



% ---------------------------------------
%   Question 2: Regular expressions (6 pnts) (10/40)
% ---------------------------------------

\clearpage

% TODO Uncomment for template
% \newpage

\begin{question}{2}\textbf{Interpreting regular expressions} (6 points)\\
\begin{enumerate}[(a)] 
\item (3 points) Given the following regular expression, provide: 
\begin{enumerate}[(i)] 
\item an example of an accepted word and 
\item an example of a rejected word, and 
\item a SHORT description, in English, of the language that this expression generates
\end{enumerate}
$$(101)(00+11)^{*}(010)$$

    \begin{solution}
       Your solution here
    \end{solution}


\item (3 points) Consider the following language on the alphabet $\Sigma = \{0,1\}$:
\begin{align*}
    L_2 = \{w \ | \ w &\texttt{ begins with any two symbols and }\\
    &\texttt{ends with the opposing two symbols (to the starting two symbols)}\}
\end{align*}
% $$L_2 = \{w \ | \ w \texttt{ begins with any two symbols and ends with the opposing two symbols (to the starting two symbols)}\}$$
(Note: the opposite of a 1 is a 0 and the opposite of a 0 is a 1; the opposite of 01 is 10)

Write a regular expression for $L$


    \begin{solution}
       Your solution here
    \end{solution}
    
\end{enumerate}
\end{question}

\newpage


% ---------------------------------------
%   Question 3: Interpreting Finite Automata (6 pnts) (16/40)
% ---------------------------------------

\begin{question}{3}\textbf{Interpreting  Finite Automata} (6 points)\\\\
Consider the following finite automaton (assume $\Sigma = \{0,1\}$):


\begin{center}
\begin{tikzpicture}[scale=0.2]
\tikzstyle{every node}+=[inner sep=0pt]
\draw [black] (35.1,-20.9) circle (3);
\draw (35.1,-20.9) node {$B$};
\draw [black] (24.2,-20.9) circle (3);
\draw (24.2,-20.9) node {$A$};
\draw [black] (24.2,-20.9) circle (2.4);
\draw [black] (44.6,-20.9) circle (3);
\draw (44.6,-20.9) node {$D$};
\draw [black] (44.6,-20.9) circle (2.4);
\draw [black] (56.1,-20.9) circle (3);
\draw (56.1,-20.9) node {$E$};
\draw [black] (39.7,-8) circle (3);
\draw (39.7,-8) node {$C$};
\draw [black] (26.165,-18.673) arc (124.58014:55.41986:6.14);
\fill [black] (26.17,-18.67) -- (27.11,-18.63) -- (26.54,-17.81);
\draw (29.65,-17.09) node [above] {$1$};
\draw [black] (22.877,-18.22) arc (234:-54:2.25);
\draw (24.2,-13.65) node [above] {$0$};
\fill [black] (25.52,-18.22) -- (26.4,-17.87) -- (25.59,-17.28);
\draw [black] (32.903,-22.906) arc (-60.57917:-119.42083:6.623);
\fill [black] (32.9,-22.91) -- (31.96,-22.86) -- (32.45,-23.73);
\draw (29.65,-24.26) node [below] {$1$};
\draw [black] (47.038,-19.183) arc (114.42694:65.57306:8.008);
\fill [black] (53.66,-19.18) -- (53.14,-18.4) -- (52.73,-19.31);
\draw (50.35,-17.97) node [above] {$0$};
\draw [black] (53.457,-22.293) arc (-71.14643:-108.85357:9.614);
\fill [black] (47.24,-22.29) -- (47.84,-23.02) -- (48.16,-22.08);
\draw (50.35,-23.31) node [below] {$0$};
\draw [black] (33.6,-8) -- (36.7,-8);
\draw (33.1,-8) node [left] {$S$};
\fill [black] (36.7,-8) -- (35.9,-7.5) -- (35.9,-8.5);
\draw [black] (38.69,-10.83) -- (36.11,-18.07);
\fill [black] (36.11,-18.07) -- (36.85,-17.49) -- (35.91,-17.15);
\draw (36.64,-13.68) node [left] {$\epsilon$};
\draw [black] (40.77,-10.8) -- (43.53,-18.1);
\fill [black] (43.53,-18.1) -- (43.72,-17.17) -- (42.78,-17.53);
\draw (42.9,-13.62) node [right] {$\epsilon$};
\draw [black] (54.777,-18.22) arc (234:-54:2.25);
\draw (56.1,-13.65) node [above] {$1$};
\fill [black] (57.42,-18.22) -- (58.3,-17.87) -- (57.49,-17.28);
\draw [black] (36.423,-23.58) arc (54:-234:2.25);
\draw (35.1,-28.15) node [below] {$0$};
\fill [black] (33.78,-23.58) -- (32.9,-23.93) -- (33.71,-24.52);
\draw [black] (45.923,-23.58) arc (54:-234:2.25);
\draw (44.6,-28.15) node [below] {$1$};
\fill [black] (43.28,-23.58) -- (42.4,-23.93) -- (43.21,-24.52);
\end{tikzpicture}
\end{center}


\begin{enumerate}[(a)]
	\item (1 point) What is the start state?
\\  \;
    \begin{solution}
       Your solution here
    \end{solution}
	\item (1 point) What is the set of accepting states?
\\  \;
    \begin{solution}
       Your solution here
    \end{solution}
	\item (1 point) Does this FA have Non-deterministic transitions? (If so, where?)
\\  \;
    \begin{solution}
       Your solution here
    \end{solution}
	\item (3 points) What is the language accepted by this FA (you may give a regular expression)?
\\  \;
    \begin{solution}
       Your solution here
    \end{solution}

\end{enumerate}
\end{question}

\clearpage

% ---------------------------------------
%   Question 4: Building Finite Automata (6 pnts) (22/40)
% ---------------------------------------
\begin{question}{4}\textbf{Building Finite Automata} (6 points)\\\\
Draw the transition diagram for a finite automaton that recognizes each of the following languages. In all cases, the alphabet is $\Sigma = \{0,1\}$.

Tip: if stuck, try to design a Regular Expression and then figure out how to draw the FA. \textbf{Note that you are allowed to use NFAs!}

Note: in the following questions, the opposite of a 1 is a 0 and the opposite of a 0 is a 1.
\begin{enumerate}[(a)]
    \item (3 points)
        \begin{align*}
            L_4 = &\{ w \in \Sigma^* \ | \ w \texttt{ begins and ends with the opposite symbol, and} \\ 
            &\texttt{ it contains the substring } 111 \}            
        \end{align*}
        

    \begin{solution}
       Your solution here
    \end{solution}
	
	\item (3 points) 

 \begin{align*}
            \{ w \in \Sigma^* \ | \ w \texttt{  does not contain the substring } 010 \}
        \end{align*}

    \begin{solution}
       Your solution here
    \end{solution}

\end{enumerate}
\end{question}

% TODO Uncomment for template
% \newpage


% ---------------------------------------
%   Question 5: Short proofs (4 pnts) (26/40)
% ---------------------------------------
\clearpage

\begin{question}{5}\textbf{Short proofs} (4 points)\\\\
Determine whether each of the following statements is \texttt{true} or \texttt{false}.\\ If it is \texttt{true}, provide a short proof. If it is \texttt{false}, give a counterexample.
\begin{enumerate}[(a)]

 \item Any non-empty Non-Regular language $L_{A}$ has a subset language $L_{B}$ such that $L_{B}$ is a regular language. 

     \begin{solution}
       Your solution here
    \end{solution}


\item The Intersection of the complement of two regular languages is not regular.

    \begin{solution}
       Your solution here
    \end{solution}


\item There exists a finite automaton that rejects the empty string.

    \begin{solution}
       Your solution here
    \end{solution}


\item There is no regular expression that can match the language:
$$ L=\{ w | \texttt{ the size of w is a multiple of four } \}$$.
 \\ \; 

     \begin{solution}
       Your solution here
    \end{solution}
    
\end{enumerate}
\end{question}


\newpage


% ---------------------------------------
%   Question 6: Non-Regular Languages (6 pnts) (32/40)
% ---------------------------------------

\begin{question}{6}\textbf{Non-Regular Languages} (6 points)\\\\
Prove that the language $L_{1x2}$ is NOT a regular language:
    \begin{align*}
        &L_{1x2} = \\
        &\{ w \in \Sigma^* \ \vert \;  \ w  \texttt{ has: } \\
        &\texttt{ zero or more 1s, followed by twice as many zeros}
        \}\\
    \end{align*}

Example accepted words: $\epsilon$, 100, 110000, 111000000, $\dots$ \\
Example rejected words: 0, 1, 10, 110, 101000, $\dots$

     \begin{solution}
       Your solution here
    \end{solution}

\end{question}

\newpage


% ---------------------------------------
%   Question 7: Context-Free Languages (6 pnts) (38/40)
% ---------------------------------------
\clearpage

\begin{question}{7}\textbf{Context-Free Languages} (6 points)\\\\
%Prove that the language:
%\[DOUBLEZERO= \{w \ | \ w \texttt{ contains twice as many } 0s \texttt{ as 1s}\}\] is context-free.
Prove that the language $L_{1x2}$ is Context-Free by 

\begin{enumerate}[i]
    \item showing its Context-Free-Grammar AND
    \item a Push-Down-Automaton $P$ such that: $L(P) = L_{1x2}$
\end{enumerate}

    \begin{align*}
        &L_{1x2} = \\
        &\{ w \in \Sigma^* \ \vert \;  \ w  \texttt{ has: } \\
        &\texttt{ zero or more 1s, followed by twice as many zeros}
        \}\\
    \end{align*}

Example accepted words: $\epsilon$, 100, 110000, 111000000, $\dots$ \\
Example rejected words: 0, 1, 10, 110, 101000, $\dots$

     \begin{solution}
       Your solution here
    \end{solution}


\end{question}
	

\newpage


% ---------------------------------------
%   Question 8: Turing Machines (2 pnts) (40/40)
% ---------------------------------------
\begin{question}{8}\textbf{Turing Machines} (2 points)\\\\
Prove that the language $L_{1x2}$ is decidable by a Turing machine:


    \begin{align*}
        &L_{1x2} = \\
        &\{ w \in \Sigma^* \ \vert \;  \ w  \texttt{ has: } \\
        &\texttt{ zero or more 1s, followed by twice as many zeros}
        \}\\
    \end{align*}

Example accepted words: $\epsilon$, 100, 110000, 111000000, $\dots$ \\
Example rejected words: 0, 1, 10, 110, 101000, $\dots$

     \begin{solution}
       Your solution here
    \end{solution}


\end{question}

\clearpage

\begin{center}
\vspace{-3em}
\textit{This page intentionally left blank as scratch paper.}
\end{center}


% --------------------------------------------------------------
%     You don't have to mess with anything below this line.
% --------------------------------------------------------------
 
\end{document}